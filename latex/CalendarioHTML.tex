\documentclass[12pt]{article}

\usepackage[spanish,es-tabla]{babel}
\usepackage{caption}
\captionsetup{labelsep=period}
\spanishdecimal{.}
\usepackage[utf8]{inputenc}
\usepackage{helvet}
\usepackage{listings}
\usepackage{xcolor}
\usepackage{apalike}
\usepackage{graphicx}
\usepackage{upquote}
\usepackage[
  colorlinks   = true, %Colours links instead of ugly boxes
  urlcolor     = blue, %Colour for external hyperlinks
  linkcolor    = blue, %Colour of internal links
  citecolor   = red, %Colour of citations
	bookmarks   = true, %Marca del ind?ce 
]{hyperref}
	
\definecolor{editorGray}{rgb}{0.95, 0.95, 0.95}
\definecolor{editorOcher}{rgb}{1, 0.5, 0} % #FF7F00 -> rgb(239, 169, 0)
\definecolor{editorGreen}{rgb}{0, 0.5, 0} % #007C00 -> rgb(0, 124, 0)

\definecolor{lightgray}{rgb}{0.95, 0.95, 0.95}
\definecolor{darkgray}{rgb}{0.4, 0.4, 0.4}
%\definecolor{purple}{rgb}{0.65, 0.12, 0.82}
\definecolor{editorGray}{rgb}{0.95, 0.95, 0.95}
\definecolor{editorOcher}{rgb}{1, 0.5, 0} % #FF7F00 -> rgb(239, 169, 0)
\definecolor{editorGreen}{rgb}{0, 0.5, 0} % #007C00 -> rgb(0, 124, 0)
\definecolor{orange}{rgb}{1,0.45,0.13}		
\definecolor{olive}{rgb}{0.17,0.59,0.20}
\definecolor{brown}{rgb}{0.69,0.31,0.31}
\definecolor{purple}{rgb}{0.38,0.18,0.81}
\definecolor{lightblue}{rgb}{0.1,0.57,0.7}
\definecolor{lightred}{rgb}{1,0.4,0.5}
\lstdefinelanguage{JavaScript}{
  morekeywords={typeof, new, true, false, catch, function, return, null, catch, switch, var, if, in, while, do, else, case, break},
  morecomment=[s]{/*}{*/},
  morecomment=[l]//,
  morestring=[b]",
  morestring=[b]'
}

%% CSS
\lstdefinelanguage{CSS}{
  morekeywords={accelerator,align-items,azimuth,background,background-attachment,
    background-color,background-image,background-position,
    background-position-x,background-position-y,background-repeat,
    behavior,border,border-bottom,border-bottom-color,
    border-bottom-style,border-bottom-width,border-collapse,
    border-color,border-left,border-left-color,border-left-style,
    border-left-width,border-right,border-right-color,
    border-right-style,border-right-width,border-spacing,
    border-style,border-top,border-top-color,border-top-style,
    border-top-width,border-width,bottom,caption-side,clear,
    clip,color,content,counter-increment,counter-reset,cue,
    cue-after,cue-before,cursor,direction,display,elevation,
    empty-cells,filter,flex-direction,float,font,font-family,font-size,
    font-size-adjust,font-stretch,font-style,font-variant,
    font-weight,justify-content,height,ime-mode,include-source,
    layer-background-color,layer-background-image,layout-flow,
    layout-grid,layout-grid-char,layout-grid-char-spacing,
    layout-grid-line,layout-grid-mode,layout-grid-type,left,
    letter-spacing,line-break,line-height,list-style,
    list-style-image,list-style-position,list-style-type,margin,
    margin-bottom,margin-left,margin-right,margin-top,
    marker-offset,marks,max-height,max-width,min-height,
    min-width,-moz-binding,-moz-border-radius,
    -moz-border-radius-topleft,-moz-border-radius-topright,
    -moz-border-radius-bottomright,-moz-border-radius-bottomleft,
    -moz-border-top-colors,-moz-border-right-colors,
    -moz-border-bottom-colors,-moz-border-left-colors,-moz-opacity,
    -moz-outline,-moz-outline-color,-moz-outline-style,
    -moz-outline-width,-moz-user-focus,-moz-user-input,
    -moz-user-modify,-moz-user-select,orphans,outline,
    outline-color,outline-style,outline-width,overflow,
    overflow-X,overflow-Y,padding,padding-bottom,padding-left,
    padding-right,padding-top,page,page-break-after,
    page-break-before,page-break-inside,pause,pause-after,
    pause-before,pitch,pitch-range,play-during,position,quotes,
    -replace,richness,right,ruby-align,ruby-overhang,
    ruby-position,-set-link-source,size,speak,speak-header,
    speak-numeral,speak-punctuation,speech-rate,stress,
    scrollbar-arrow-color,scrollbar-base-color,
    scrollbar-dark-shadow-color,scrollbar-face-color,
    scrollbar-highlight-color,scrollbar-shadow-color,
    scrollbar-3d-light-color,scrollbar-track-color,table-layout,
    text-align,text-align-last,text-decoration,text-indent,
    text-justify,text-overflow,text-shadow,text-transform,
    text-autospace,text-kashida-space,text-underline-position,top,
    unicode-bidi,-use-link-source,vertical-align,visibility,
    voice-family,volume,white-space,widows,width,word-break,
    word-spacing,word-wrap,writing-mode,z-index,zoom},
  morestring=[s]{:}{;},
  sensitive,
  morecomment=[s]{/*}{*/}
}

% JavaScript
\lstdefinelanguage{JavaScript}{
  morekeywords={typeof, new, true, false, catch, function, return, null, catch, switch, var, if, in, while, do, else, case, break,let, const},
  morecomment=[s]{/*}{*/},
  morecomment=[l]//,
  morestring=[b]",
  morestring=[b]'
}

\lstdefinelanguage{HTML5}{
  language=html,
  sensitive=true,	
  alsoletter={<>=-},	
  morecomment=[s]{<!-}{-->},
  tag=[s],
  otherkeywords={
  % General
  >,
  % Standard tags
	<!DOCTYPE,
  </html, <html, <head, <title, </title, <style, </style, <link, </head, <meta, />,
	% body
	</body, <body,
	% Divs
	</div, <div, </div>, 
	% Paragraphs
	</p, <p, </p>,
	% scripts
	</script, <script,
  % More tags...
  <canvas, /canvas>, <svg, <rect, <animateTransform, </rect>, </svg>, <video, <source, <iframe, </iframe>, </video>, <image, </image>, <header, </header, <article, </article
  },
  ndkeywords={
  % General
  =,
  % HTML attributes
  charset=, src=, id=, width=, height=, style=, type=, rel=, href=,
  % SVG attributes
  fill=, attributeName=, begin=, dur=, from=, to=, poster=, controls=, x=, y=, repeatCount=, xlink:href=,
  % properties
  margin:, padding:, background-image:, border:, top:, left:, position:, width:, height:, margin-top:, margin-bottom:, font-size:, line-height:,
	% CSS3 properties
  transform:, -moz-transform:, -webkit-transform:,
  animation:, -webkit-animation:,
  transition:,  transition-duration:, transition-property:, transition-timing-function:,
  }
}

\lstdefinestyle{htmlcssjs} {%
  % General design
  backgroundcolor=\color{editorGray},
  basicstyle={\footnotesize\ttfamily},   
  frame=single,
  % line-numbers
  xleftmargin={0.75cm},
  numbers=left,
  stepnumber=1,
  firstnumber=1,
  numberfirstline=true,	
  % Code design
  identifierstyle=\color{black},
  keywordstyle=\color{blue}\bfseries,
  ndkeywordstyle=\color{editorGreen}\bfseries,
  stringstyle=\color{editorOcher}\ttfamily,
  commentstyle=\color{brown}\ttfamily,
  % Code
  language=HTML5,
  alsolanguage=JavaScript,
  alsodigit={.:;},	
  tabsize=2,
  showtabs=false,
  showspaces=false,
  showstringspaces=false,
  extendedchars=true,
  breaklines=true,
	literate=%
	%Acentos
  {�}{\'{a}}1
  {�}{\'{e}}1
  {�}{\'{i}}1
  {�}{\'{o}}1
  {�}{\'{u}}1
	{�}{\'{A}}1
  {�}{\'{E}}1
  {�}{\'{I}}1
  {�}{\'{O}}1
  {�}{\'{U}}1
}

\lstdefinestyle{css} {
	% General design
    backgroundcolor=\color{lightgray},
    basicstyle={\small\ttfamily},   
    frame=l,
    % Code design
    identifierstyle=\color{black},
    keywordstyle=\color{blue}\bfseries,
    ndkeywordstyle=\color{greenCode}\bfseries,
    stringstyle=\color{editorOcher}\ttfamily,
    commentstyle=\color{darkgray}\ttfamily,
    % Code
    language={CSS},
    tabsize=2,
    showtabs=false,
    showspaces=false,
    showstringspaces=false,
    extendedchars=true,
    breaklines=true,
    % line-numbers
    xleftmargin={0.75cm},
    numbers=left,
    stepnumber=1,
    firstnumber=1,
    numberfirstline=true,
}
%
\lstdefinestyle{py} {%
language=python,
literate=%
*{0}{{{\color{lightred}0}}}1
{1}{{{\color{lightred}1}}}1
{2}{{{\color{lightred}2}}}1
{3}{{{\color{lightred}3}}}1
{4}{{{\color{lightred}4}}}1
{5}{{{\color{lightred}5}}}1
{6}{{{\color{lightred}6}}}1
{7}{{{\color{lightred}7}}}1
{8}{{{\color{lightred}8}}}1
{9}{{{\color{lightred}9}}}1,
basicstyle=\footnotesize\ttfamily, % Standardschrift
numbers=left,               % Ort der Zeilennummern
%numberstyle=\tiny,          % Stil der Zeilennummern
%stepnumber=2,               % Abstand zwischen den Zeilennummern
numbersep=5pt,              % Abstand der Nummern zum Text
tabsize=4,                  % Groesse von Tabs
extendedchars=true,         %
breaklines=true,            % Zeilen werden Umgebrochen
keywordstyle=\color{blue}\bfseries,
frame=single,
commentstyle=\color{brown}\itshape,
stringstyle=\color{editorOcher}\ttfamily, % Farbe der String
showspaces=false,           % Leerzeichen anzeigen ?
showtabs=false,             % Tabs anzeigen ?
xleftmargin=17pt,
framexleftmargin=17pt,
framexrightmargin=5pt,
framexbottommargin=4pt,
%backgroundcolor=\color{lightgray},
showstringspaces=false,      % Leerzeichen in Strings anzeigen ?
backgroundcolor=\color{editorGray},
}%
%
\makeatother


%%%%TIKZ
\usepackage{tikz}
\usetikzlibrary{shapes,arrows}

% Define block styles
\tikzstyle{decision} = [diamond, draw, fill=blue!20, 
    text width=4.5em, text badly centered, node distance=3cm, inner sep=0pt]
\tikzstyle{block} = [rectangle, draw, fill=blue!20, 
    text width=7em, text centered, rounded corners, minimum height=4em]
\tikzstyle{line} = [draw, -latex']
\tikzstyle{cloud} = [draw, ellipse,fill=red!20, node distance=3cm,
    minimum height=2em]



\title{Calendario mediante HTML}
\author{Isaac Rosales García }
\date{\today}

\begin{document}
\maketitle
\tableofcontents

\section{Introducción}
En este trabajo se muestran los diagramas de flujos para ver el comportamiento de las funciones usadas para el uso del calendario

\section{HTML}
La base de HTML es la que se muestra a continuación

%\resizebox{.5\textwidth}{!}{
\begin{lstlisting}[style=htmlcssjs][frame=single]
<!DOCTYPE html>
<html>
  <head>
    <link rel="stylesheet" href="assets/index.css" />

    <title>Calendario</title>
  </head>

  <body>
    <select id="month" onchange="changeMonth()"></select>
    <div id="miModal" class="modal">
      <div class="modalContent">
        <div>
          <span class="close" onclick="closeModal()"> &times; </span>
          <h2>Agregar Evento</h3>
        </div>
        <p class="modalForm modalDate">Fecha</p>
        <form id="calendarForm" class="modalForm">
          <label>Titulo</label><br />
          <input type="text" name="title" required /><br />
          <label>Descripcion</label><br />
          <input type="text" name="description" required/><br />
          <label>Participantes</label><br />
          <input type="text" name="people"/><br /><br />
          <div>
            <input type="submit" value="Guardar" class="modalForm modalButton" />
          </div>
        </form>
      </div>
    </div>
    <div id="container">
      <div>
        <h1 id="monthDate">Junio 2020</h1>
      </div>
       <div class="calendarContainerHeader">
        <div>Lunes</div>
        <div>Martes</div>
        <div>Miercoles</div>
        <div>Jueves</div>
        <div>Viernes</div>
        <div>Sabado</div>
        <div>Domingo</div>
      </div>
    </div>
    <hr>
    <div id="dynamicHTML">
    </div>
    <script type="application/javascript" src="assets/index.js"></script>
  </body>
</html>
\end{lstlisting}
%}

\section{css}

A continuación se muestran las clases

\begin{lstlisting}[style=css][frame=single]
#container,
#dynamicHTML {
  display: flex;
  flex-direction: column;
  align-items: center;
  height: 100vh;
}

.calendarContainer {
  display: flex;
  flex-direction: row;
  justify-content: space-evenly;
  text-align: center;
}

.calendarContainerHeader {
  display: flex;
  flex-direction: row;
  justify-content: space-evenly;
  text-align: center;
}

.calendarContainer > div {
  height: 5rem;
  width: 5rem;
  background-color: white;
  border: 1px solid #cdcdcd;
  display: flex;
  flex-direction: column-reverse;
  justify-content: flex-end;
  align-items: flex-end;
  margin: 0px;
  padding: 5px;
  font-weight: bold;
}

.calendarContainerHeader > div {
  height: 1rem;
  width: 5rem;
  background-color: #d1d1d1;
  border: 1px solid #bbbbbb;
  display: flex;
  justify-content: space-evenly;
  align-items: center;
  margin: 0px;
  padding: 5px;
  font-weight: bold;
}

.dayClass {
  color: #474b4e;
}

.lastMonth {
  color: #bbbbbb;
}

.modal {
  display: none;
  position: fixed;
  z-index: 99999;
  width: 100vw;
  height: 100vh;
  background-color: rgba(0, 0, 0, 0.3);
  left: 0;
  top: 0;
  padding-top: 100px;
  overflow: hidden;
}

.modalContent {
  display: flex;
  flex-direction: column;
  justify-content: space-evenly;
  text-align: center;
  background-color: #ebebeb;
  margin: auto;
  padding: 20px;
  border: 1px solid #888;
  width: 25em;
  height: 28em;
}

.modalForm {
  display: flex;
  flex-direction: column;
  justify-content: space-evenly;
  text-align: left;
  margin-left: 60px;
  margin-right: 60px;
  font-weight: bold;
  background-color: #ebebeb;
}

.modalDate {
  display: flex;
  flex-direction: column;
  justify-content: space-evenly;
  text-align: right;
  margin-right: 60px;
  background-color: #ebebeb;
}

.modalButton {
  flex-direction: row;
  background-color: #595959;
  color: white;
  width: 8em;
  height: 2em;
  margin-left: 85px;
  border-radius: 15px;
}

.close {
  color: #aaa;
  float: right;
  font-size: 28px;
  font-weight: bold;
}

.close:hover,
.close:focus {
  cursor: pointer;
  color: #000;
  text-decoration: none;
}

.calendarContainer > div:hover {
  background-color: #e3e3e3;
  transition-delay: 50ms;
  cursor: pointer;
}

.descripcion {
  background-color: #24f0b6;
  margin: auto;
  padding: 3px;
  border: 1px solid #117055;
  border-radius: 10px;
  box-shadow: -2px -2px 5px #32705f;
  width: 4em;
  height: 1em;
}

.descripcion:hover {
  background-color: #21d9a5;
  transition-delay: 50ms;
  cursor: pointer;
}

\end{lstlisting}
\section{Creación de días}
El objetivo de esta función consiste en crear un elemento div

\begin{lstlisting}[style=htmlcssjs][frame=single]
let createDays = () => {

  let container = document.getElementById("container");

  for (const weekEl in dayObjectBox) {
    let div = document.createElement("div");
    div.className = "calendarContainer";
    container.appendChild(div);

    dayObjectBox[weekEl].forEach((dayEl) => {
      let divDay = document.createElement("div");
      divDay.dataset.position = dayEl;
      div.appendChild(divDay);
      divDay.className = "dayClass";
    });
  }
};
\end{lstlisting}

\resizebox{.5\textwidth}{!}{
\begin{tikzpicture}[node distance = 2.5cm]
    % Place nodes
    \node [block] (init) {iniciar función};
    \node [block, below of=init] (getElement) {Se obtiene el elemento HTML <<container>>};
    \node [block, below of=getElement] (initFor) {Ciclo for para elemento de <<dayObjectBox>>};
    \node [block, below of=initFor] (createDiv) {Crea un elemento div};
		\node [block, left of=createDiv, node distance=4cm] (stepOn) {Pasa por ciclo de weekEl};
		\node [cloud, below of=createDiv, node distance=2cm] (stop) {fin};
    % Draw edges
    \path [line] (init) -- (getElement);
    \path [line] (getElement) -- (initFor);
    \path [line] (initFor) -- (createDiv);
		\path [line] (createDiv) -- (stepOn);
    \path [line] (stepOn) |- node[yshift=0.2cm] {yes}(initFor);
		\path [line] (createDiv) -- (stop);
    %\path [line] (update) |- (identify);
    %\path [line] (decide) -- node {no}(stop);
    %\path [line,dashed] (expert) -- (init);
    %\path [line,dashed] (system) -- (init);
    %\path [line,dashed] (system) |- (evaluate);
\end{tikzpicture}
}
\end{document}